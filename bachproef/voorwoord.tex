%%=============================================================================
%% Voorwoord
%%=============================================================================

\chapter*{\IfLanguageName{dutch}{Woord vooraf}{Preface}}%
\label{ch:voorwoord}

%% TODO:
%% Het voorwoord is het enige deel van de bachelorproef waar je vanuit je
%% eigen standpunt (``ik-vorm'') mag schrijven. Je kan hier bv. motiveren
%% waarom jij het onderwerp wil bespreken.
%% Vergeet ook niet te bedanken wie je geholpen/gesteund/... heeft






Voor u ligt mijn bachelorproef getiteld "Secondary Babytalkmonitor: Vergelijken van verschillende Speech-to-Text Modellen". Dit essentiële onderdeel van mijn academische traject vormt de sluitsteen van mijn studie Toegepaste Informatica aan Hogeschool Gent. Deze scriptie is niet alleen het einde van een belangrijke etappe, maar ook de start van een spannende nieuwe fase in mijn professionele carrière in België als international student. Mijn avontuur begon met de uitdaging om de Nederlandse taal te leren, wat de basis legde voor mijn studies en mijn toekomstige professionele ontwikkeling in dit land. Deze reis heeft mij enorm geïnspireerd en voorbereid op de volgende stappen.

Ik wil graag mijn oprechte dank uitspreken aan mijn promotor Anneleen Bekkens en co-promotor El Hassan Baazizi, wiens deskundige begeleiding en onwankelbare steun van onschatbare waarde zijn geweest voor het bereiken van dit moment. Mijn dankbaarheid gaat ook uit naar mijn vrienden en familie, die mij door dik en dun hebben gesteund. Daarnaast wil ik de vrijwilligers niet vergeten, die essentiële bijdragen hebben geleverd aan het verzamelen van gegevens voor mijn onderzoek. Zonder hun hulp zou dit project niet tot stand zijn gekomen.

Al vanaf jonge leeftijd was ik gefascineerd door het oplossen van puzzels, een hobby die mij natuurlijk naar de wereld van de technologie en programmering leidde. Deze interesse evolueerde naar een diepe passie voor de computerwetenschappen, wat uiteindelijk resulteerde in een bijzondere aandacht voor de mogelijkheden van kunstmatige intelligentie (AI). Binnen het domein van AI heb ik een specifieke interesse ontwikkeld voor Automatic Speech Recognition (ASR) technologieën. Het gebruik van deze geavanceerde technologieën in de gezondheidszorg biedt aanzienlijke mogelijkheden voor het verbeteren van de communicatie tussen zorgverleners en oudere patiënten. Dit potentieel strekt zich uit tot diverse aspecten van de gezondheidszorg, van diagnose en behandeling tot dagelijkse zorg en preventie.
\\
Dit onderzoek richt zich op het evalueren van verschillende speech-to-text modellen om te bepalen welke het meest geschikt zijn voor het identificeren en monitoren van 'secondary babytalk' of 'elderspeak', een specifieke communicatiestijl gebruikt in de interactie met oudere mensen. Door het meest effectieve model te selecteren, streven we ernaar de communicatie binnen de zorgsector te verbeteren en daarmee de algehele kwaliteit van zorg te verhogen.
\\
Terwijl u door deze pagina's bladert, hoop ik dat u inzicht verkrijgt in de cruciale rol die ASR-technologieën kunnen spelen in de verbetering van onze interacties en zorgpraktijken, en in hoe geavanceerde technologie kan bijdragen aan een empathischere en effectievere communicatie.
\\
Ik wens u veel leesplezier en dank u voor uw interesse in mijn werk.
\\
\\
Ayoub Ait Cheikh Ahmed.
\\
Antwerpen 21/04/2024