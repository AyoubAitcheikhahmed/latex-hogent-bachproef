%%=============================================================================
%% Voorwoord
%%=============================================================================

\chapter*{\IfLanguageName{dutch}{Woord vooraf}{Preface}}%
\label{ch:voorwoord}

%% TODO:
%% Het voorwoord is het enige deel van de bachelorproef waar je vanuit je
%% eigen standpunt (``ik-vorm'') mag schrijven. Je kan hier bv. motiveren
%% waarom jij het onderwerp wil bespreken.
%% Vergeet ook niet te bedanken wie je geholpen/gesteund/... heeft


 bachelorproef markeert een significante mijlpaal in mijn academische reis aan de Hogeschool Gent, waar ik de kans kreeg om me te specialiseren in Systeem- en Netwerkbeheer binnen de opleiding Bachelor in de Toegepaste Informatica. Mijn fascinatie voor de dynamische wereld van telemedicine, versterkt door de onvoorziene uitdagingen en kansen gepresenteerd door de COVID-19 pandemie, heeft mij geïnspireerd om de implementatiemethoden van webarchitecturen binnen Amazon Web Services (AWS) te verkennen. Deze keuze weerspiegelt niet alleen mijn academische en professionele interesses maar ook de relevante vraagstukken in een tijd waarin digitale gezondheidszorgoplossingen essentieel zijn geworden.

Door mijn stage als Cloud Engineer heb ik waardevolle inzichten verkregen in de complexiteiten van cloud computing en de cruciale balans tussen handmatige en geautomatiseerde implementatieprocessen. Dit onderzoek richt zich op het snijvlak van mijn academische curiositeit en professionele ambities, met een specifieke focus op de implementatie van cloud-gebaseerde oplossingen in de telemedicine sector, een domein dat aanzienlijk is gegroeid onder de invloed van de wereldwijde pandemie.

Ik wil graag mijn oprechte dankbaarheid uitspreken aan mijn promotor, Gilles Blondeel, en mijn co-promotor, Sander Rademaker. Hun deskundige begeleiding, geduld, en aanmoediging hebben een fundamentele rol gespeeld in mijn onderzoeksproces. Hun inzichten en steun waren cruciaal in het navigeren door de complexe vraagstukken rondom telemedicine en cloud-gebaseerde technologieën.

Mijn dankbaarheid gaat ook uit naar mijn familie en vrienden, wiens onvoorwaardelijke steun en geloof in mijn capaciteiten mij de kracht en motivatie hebben gegeven om deze uitdaging aan te gaan. Het balanceren van deze diepgaande studie met persoonlijke en professionele verplichtingen was geen eenvoudige opgave. Hun steun was de drijvende kracht die mij door dit proces heeft geloodst.

Deze proef is het culminatiepunt van een diepgaande academische verkenning en persoonlijke groei. Het weerspiegelt mijn streven naar kennis en mijn ambitie om een positieve bijdrage te leveren aan de wereld van telemedicine en cloud computing. Ik hoop dat dit onderzoek zal dienen als een waardevolle bron voor toekomstige studenten en professionals die geïnteresseerd zijn in het kruispunt van gezondheidszorg en technologie.
Deze proef is het culminatiepunt van een diepgaande academische verkenning en persoonlijke groei. Het weerspiegelt mijn streven naar kennis en mijn ambitie om een positieve bijdrage te leveren aan de wereld van telemedicine en cloud computing. Ik hoop dat dit onderzoek zal dienen als een waardevolle bron voor toekomstige studenten en professionals die geïnteresseerd zijn in het kruispunt van gezondheidszorg en technologie.


    