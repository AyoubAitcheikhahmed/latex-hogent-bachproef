%%=============================================================================
%% Methodologie
%%=============================================================================

\chapter{\IfLanguageName{dutch}{Methodologie}{Methodology}}%
\label{ch:methodologie}

Dit hoofdstuk biedt een gestructureerd overzicht van de methodologische aanpak die is gehanteerd in dit onderzoek. Het is bedoeld om de lezer inzicht te geven in de onderzoeksopzet, de gehanteerde methoden en de rationale achter de gekozen aanpak. Het onderzoek is opgedeeld in verschillende fasen, elk met specifieke doelstellingen, deliverables, en onderzoeksmethoden. De onderstaande secties beschrijven deze fasen in detail.

%Literatuurstudie
De eerste fase van het onderzoek bestond uit een uitgebreide literatuurstudie, gericht op het verkennen van het bestaande kennisdomein met betrekking tot 'secondary babytalk', de beschikbare spraak-naar-tekst (ASR) modellen, en de criteria voor model evaluatie. Deze fase had als doelstelling het vaststellen van een theoretische basis voor het onderzoek en het identificeren van lacunes in de bestaande literatuur. De deliverable van deze fase was een gedetailleerde literatuuroverzicht, die als fundament dient voor het verdere onderzoek.

%Selectie van ASR Modellen
Gebaseerd op de literatuurstudie, volgde een fase waarin specifieke ASR modellen werden geselecteerd voor verdere analyse. Deze selectie was gebaseerd op een long-list die werd samengesteld aan de hand van criteria zoals beschikbaarheid en taalondersteuning. Een short-list van modellen werd vervolgens geselecteerd voor benchmarking. Het gebruik van deze gestructureerde selectieprocedure zorgde voor een methodische benadering in de keuze van modellen.

%Opzetten Testomgeving
De volgende stap in het onderzoek was het opzetten van een testomgeving voor het uitvoeren van de benchmarking. Dit omvatte het ontwikkelen van een gestandaardiseerde set van spraakopnames die 'secondary babytalk' kenmerken bevatten, en het voorbereiden van de technische infrastructuur voor het testen van de geselecteerde ASR modellen. Deze fase was cruciaal om ervoor te zorgen dat de evaluatie onder gecontroleerde en reproduceerbare omstandigheden plaatsvond.

%Uitvoeren van Benchmarking
In deze fase werden de geselecteerde ASR modellen onderworpen aan een reeks tests om hun prestaties te evalueren op basis van vooraf vastgestelde criteria zoals Word Error Rate (WER) en Jaro-Winkler afstand. Deze benchmarking stelde ons in staat om kwantitatieve data te verzamelen over de effectiviteit van elk model in het accuraat transcriberen van 'secondary babytalk'.

%analyse en interppretatie 
De verzamelde data uit de benchmarking fase werden vervolgens geanalyseerd en geïnterpreteerd. Deze analyse omvatte een vergelijking van de prestaties van de verschillende ASR modellen. De doelstelling was om inzicht te krijgen in de geschiktheid van elk model voor de specifieke toepassing.

%% TODO: In dit hoofstuk geef je een korte toelichting over hoe je te werk bent
%% gegaan. Verdeel je onderzoek in grote fasen, en licht in elke fase toe wat
%% de doelstelling was, welke deliverables daar uit gekomen zijn, en welke
%% onderzoeksmethoden je daarbij toegepast hebt. Verantwoord waarom je
%% op deze manier te werk gegaan bent.
%% 
%% Voorbeelden van zulke fasen zijn: literatuurstudie, opstellen van een
%% requirements-analyse, opstellen long-list (bij vergelijkende studie),
%% selectie van geschikte tools (bij vergelijkende studie, "short-list"),
%% opzetten testopstelling/PoC, uitvoeren testen en verzamelen
%% van resultaten, analyse van resultaten, ...
%%
%% !!!!! LET OP !!!!!
%%
%% Het is uitdrukkelijk NIET de bedoeling dat je het grootste deel van de corpus
%% van je bachelorproef in dit hoofstuk verwerkt! Dit hoofdstuk is eerder een
%% kort overzicht van je plan van aanpak.
%%
%% Maak voor elke fase (behalve het literatuuronderzoek) een NIEUW HOOFDSTUK aan
%% en geef het een gepaste titel.

