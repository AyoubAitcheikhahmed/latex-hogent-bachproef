%%=============================================================================
%% Samenvatting
%%=============================================================================

% TODO: De "abstract" of samenvatting is een kernachtige (~ 1 blz. voor een
% thesis) synthese van het document.
%
% Een goede abstract biedt een kernachtig antwoord op volgende vragen:
%
% 1. Waarover gaat de bachelorproef?
% 2. Waarom heb je er over geschreven?
% 3. Hoe heb je het onderzoek uitgevoerd?
% 4. Wat waren de resultaten? Wat blijkt uit je onderzoek?
% 5. Wat betekenen je resultaten? Wat is de relevantie voor het werkveld?
%
% Daarom bestaat een abstract uit volgende componenten:
%
% - inleiding + kaderen thema
% - probleemstelling
% - (centrale) onderzoeksvraag
% - onderzoeksdoelstelling
% - methodologie
% - resultaten (beperk tot de belangrijkste, relevant voor de onderzoeksvraag)
% - conclusies, aanbevelingen, beperkingen
%
% LET OP! Een samenvatting is GEEN voorwoord!

%%---------- Nederlandse samenvatting -----------------------------------------
%
% TODO: Als je je bachelorproef in het Engels schrijft, moet je eerst een
% Nederlandse samenvatting invoegen. Haal daarvoor onderstaande code uit
% commentaar.
% Wie zijn bachelorproef in het Nederlands schrijft, kan dit negeren, de inhoud
% wordt niet in het document ingevoegd.

\IfLanguageName{english}{%
\selectlanguage{dutch}
\chapter*{Samenvatting}
\lipsum[1-4]
\selectlanguage{english}
}{}

%%---------- Samenvatting -----------------------------------------------------
% De samenvatting in de hoofdtaal van het document

\chapter*{\IfLanguageName{dutch}{Samenvatting}{Abstract}}

In dit bachelorproefonderzoek worden diverse speech-to-text modellen, inclusief open source modellen, getest om te bepalen welk model het meest geschikt is voor de use case van spraak naar tekst in een secondary babytalk monitor. Vrijwilligers verzamelen verschillende steekproeven van Vlaamse accenten door hun audio-opnames te delen, die vervolgens worden gebruikt als dataset. Deze audio samples bestaan uit voorgemaakte scripts die vaak worden gebruikt in rusthuizen en ziekenhuizen, zoals voorbereiden om te slapen en medicatie innemen. De opnames worden getranscribeerd door meerdere \gls{ai} modellen om een transcriptie te verkrijgen. Vervolgens worden de transcripties geëvalueerd met verschillende metrics zoals \gls{wer}, \gls{mer}, en \gls{wil}.

Het doel van dit onderzoek is de meest effectieve methode voor het herkennen en transcriberen van 'Secondary Babytalk', een specifieke communicatiestijl gebruikt bij interacties met oudere patiënten, te identificeren. Door nauwkeurige evaluatie streven we ernaar het beste en meest efficiënte model te selecteren.

De literatuurstudie die aan de basis ligt van dit onderzoek, biedt een diepgaand inzicht in de beschikbare ASR-modellen en de nauwkeurigheidsmetrieken waarmee hun prestaties kunnen worden beoordeeld. De belangrijkste beoordelingscriteria die gebruikt werd, is de JIWER bibliotheek die verschillende mestrieken gebruikt om een transcriptie te evalueren. De laatste helpt om de effectiviteit van elk model en de accuratie te meten.

Het onderzoek omvat een gedetailleerde analyse van verschillende \gls{asr}-modellen die momenteel op de markt beschikbaar zijn. Deze modellen worden systematisch getest en vergeleken om hun geschiktheid voor gebruik in de zorgsector te bepalen. Door het uitvoeren van een reeks benchmarktests wordt inzicht verkregen in welke modellen het beste presteren onder gevestigde evaluatiecriteria.

Daarnaast wordt in het onderzoek de impact van 'Secondary Babytalk' onderzocht, inclusief de manier waarop het vaak negatief wordt ontvangen door oudere patiënten, ondanks de positieve intenties van zorgverleners. Dit benadrukt de noodzaak voor verbeterde communicatiemethoden die respect tonen voor de autonomie en competentie van de oudere patiënt.

Op basis van de resultaten van de analyse zal dit onderzoek praktische aanbevelingen doen voor het kiezen van het meest geschikte \gls{asr}-model. Deze aanbevelingen zijn gericht op het optimaliseren van de communicatiepraktijken binnen de zorgsector en het verbeteren van de interactie tussen zorgverleners en oudere patiënten, wat uiteindelijk zal bijdragen aan een verhoogde levenskwaliteit en welzijn van deze patiëntengroep. Wat wordt ondersteund door het onderzoek van \textcite{Sibian2024}.
