%%=============================================================================
%% Samenvatting
%%=============================================================================

% TODO: De "abstract" of samenvatting is een kernachtige (~ 1 blz. voor een
% thesis) synthese van het document.
%
% Een goede abstract biedt een kernachtig antwoord op volgende vragen:
%
% 1. Waarover gaat de bachelorproef?
% 2. Waarom heb je er over geschreven?
% 3. Hoe heb je het onderzoek uitgevoerd?
% 4. Wat waren de resultaten? Wat blijkt uit je onderzoek?
% 5. Wat betekenen je resultaten? Wat is de relevantie voor het werkveld?
%
% Daarom bestaat een abstract uit volgende componenten:
%
% - inleiding + kaderen thema
% - probleemstelling
% - (centrale) onderzoeksvraag
% - onderzoeksdoelstelling
% - methodologie
% - resultaten (beperk tot de belangrijkste, relevant voor de onderzoeksvraag)
% - conclusies, aanbevelingen, beperkingen
%
% LET OP! Een samenvatting is GEEN voorwoord!

%%---------- Nederlandse samenvatting -----------------------------------------
%
% TODO: Als je je bachelorproef in het Engels schrijft, moet je eerst een
% Nederlandse samenvatting invoegen. Haal daarvoor onderstaande code uit
% commentaar.
% Wie zijn bachelorproef in het Nederlands schrijft, kan dit negeren, de inhoud
% wordt niet in het document ingevoegd.

\IfLanguageName{english}{%
\selectlanguage{dutch}
\chapter*{Samenvatting}
\lipsum[1-4]
\selectlanguage{english}
}{}

%%---------- Samenvatting -----------------------------------------------------
% De samenvatting in de hoofdtaal van het document

\chapter*{\IfLanguageName{dutch}{Samenvatting}{Abstract}}

in dit bachelorproef onderzoek worden verschillende speech-to-text modellen getest inclusief open source modellen. om achter te halen welke model het meest geschikt is voor de use case van spraak naar tekst in een secondary babytalk monitor. Er worden verschillende samples van verschillende accenten in vlanderen verzameld met behulp van vrijwillegers die hun audio opnames delen om die te gebruiken als dataset. de audio samples bestaan uit een voorgemaakte scripte die vaak gebeurt in rusthuizen en ziekenhuizen bv: (voorbereiden om te slapen, medicatie innemen). gmVervolgens worden de audio opnames getranscribeerd door meerdere AI modellen om een transcriptie te krijgen. als laatste stap worden de transcripties gevalueerd door verschillende metrics zoals Word Error Rate (WER) , Match Error Rate (MER), Word Information Lost (WIL). 

Dit onderzoek richt zich op de evaluatie en vergelijking van verschillende speech-to-text (ASR) modellen om de meest effectieve methode te identificeren voor het herkennen en transcriberen van 'Secondary Babytalk', een specifieke communicatiestijl gebruikt in de interactie met oudere patiënten. Door een naukeurig evaluatie met gebruik te maken van verschillende metrics zoals Word Error Rate (WER) kunnen we het beste en efficente model uitkiezen. Door dit evaluatie streeft dit  onderzoek ernaar de communicatie tussen zorgverleners en oudere patiënten te verbeteren, wat essentieel is voor het verhogen van de kwaliteit van zorg.

De literatuurstudie die aan de basis ligt van dit onderzoek, biedt een diepgaand inzicht in de beschikbare ASR-modellen en de nauwkeurigheidsmetrieken waarmee hun prestaties kunnen worden beoordeeld. De belangrijkste beoordelingscriteria die worden gebruikt, zijn de Word Error Rate (WER) en de Jaro-Winkler afstand (JWD), welke helpen om de effectiviteit van elk model in het identificeren van 'Secondary Babytalk' kenmerken te meten.

Het onderzoek omvat een gedetailleerde analyse van verschillende ASR-modellen die momenteel op de markt beschikbaar zijn. Deze modellen worden systematisch getest en vergeleken om hun geschiktheid voor gebruik in de zorgsector te bepalen. Door het uitvoeren van een reeks benchmarktests wordt inzicht verkregen in welke modellen het beste presteren onder gevestigde evaluatiecriteria.

Daarnaast wordt in het onderzoek de impact van 'Secondary Babytalk' onderzocht, inclusief de manier waarop het vaak negatief wordt ontvangen door oudere patiënten, ondanks de positieve intenties van zorgverleners. Dit benadrukt de noodzaak voor verbeterde communicatiemethoden die respect tonen voor de autonomie en competentie van de oudere patiënt.

Op basis van de resultaten van de analyse zal dit onderzoek praktische aanbevelingen doen voor het kiezen van het meest geschikte ASR-model. Deze aanbevelingen zijn gericht op het optimaliseren van de communicatiepraktijken binnen de zorgsector en het verbeteren van de interactie tussen zorgverleners en oudere patiënten, wat uiteindelijk zal bijdragen aan een verhoogde levenskwaliteit en welzijn van deze patiëntengroep.
