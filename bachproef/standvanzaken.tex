\chapter{\IfLanguageName{dutch}{Stand van zaken}{State of the art}}%
\label{ch:stand-van-zaken}

% Tip: Begin elk hoofdstuk met een paragraaf inleiding die beschrijft hoe
% dit hoofdstuk past binnen het geheel van de bachelorproef. Geef in het
% bijzonder aan wat de link is met het vorige en volgende hoofdstuk.

% Pas na deze inleidende paragraaf komt de eerste sectiehoofding.
\section{Inleiding}


Het hoofdstuk over literatuurstudie vormt een essentieel fundament voor dit onderzoek, omdat het een grondig begrip biedt van het bestaande kennislandschap met betrekking tot het onderwerp. Door een gedegen literatuurstudie uit te voeren, kunnen we verschillende cruciale doelen bereiken die de basis leggen voor ons onderzoek.

In dit onderzoek wil men nagaan of diverse speech-to-text modellen die momenteel beschikbaar zijn op de markt, met als primair doel het nauwkeurig identificeren van kenmerken die inherent zijn aan 'secondary babytalk'. Door het selecteren van het meest geschikte model en het uitvoeren van benchmarking, wordt beoogd de communicatie tussen zorgverleners en oudere patiënten te verbeteren. De benchmarking zal worden uitgevoerd door de prestaties van de verschillende modellen te vergelijken aan de hand van gevestigde criteria, zoals de Word Error Rate (WER) en de Jaro-Winkler afstand. 
Door middel van een gedetailleerde analyse van verschillende modellen zullen we inzicht krijgen in hun prestaties en geschiktheid voor het identificeren van 'secondary babytalk'. Dit onderzoek zal resulteren in praktische aanbevelingen voor het selecteren van het meest geschikte model om de communicatie in de zorgsector te verbeteren, wat de algehele kwaliteit van zorg ten goede zal komen.
%Deze vergelijking zal het onderzoek in staat stellen om het meest geschikte model te identificeren voor het accuraat identificeren van 'secondary babytalk' in de communicatie tussen zorgverleners en oudere patiënten.


Alvoerens hierop een antwoord te kunnen bieden, is het van belang inzicht te verwerven in wat er vergeleken dient te worden. Dit kunnen we bewerkstelligen door te antwoorden op de volgende vragen:
\begin{enumerate}[label=\arabic*.]
    \item \textbf{Definitie en Plaatsing van Speech-to-Text Modellen:}
    \begin{itemize}
        \item Wat omvat precies een spraak-naar-tekstmodel en waar situeert het zich binnen het domein van kunstmatige intelligentie?
        \item Hoe past een spraak-naar-tekstmodel in het bredere landschap van AI en wat zijn de kernprincipes achter deze technologie?
    \end{itemize}
    
    \item \textbf{Beschikbare Modellen op de Markt:}
    \begin{itemize}
        \item Welke specifieke spraak-naar-tekstmodellen zijn momenteel beschikbaar op de markt?
        \item Hoe verschillen deze modellen in accuratie?
    \end{itemize}
    
    \item \textbf{Relevante Criteria voor Model Evaluatie:}
    \begin{itemize}
        \item Welke criteria zijn van belang bij het evalueren van spraak-naar-tekstmodellen?
        \item Hoe kunnen we de prestaties meten aan de hand van zowel kwantitatieve als kwalitatieve aspecten?
        \item Welke overwegingen zijn cruciaal voor het afstemmen van een model op de specifieke usecase van spraaktranscriptie?
    \end{itemize}
\end{enumerate}

\section{Vergrijzing in Vlaanderen}
-Een analyse van de demografische evolutie en de gevolgen van de babyboom in de jaren '60. 

- Toekomstige uitdagingen in de zorgsector door de vergrijzing. 

In de hedendaagse zorg- en welzijnssector is er een verschuiving zichtbaar van aanbodgestuurde naar persoonsgerichte zorg. Dit houdt in dat zorgverleners zorg op maat bieden, rekening houdend met de unieke behoeften, voorkeuren, wensen en verwachtingen van elke cliënt of patiënt. Persoonsgerichte zorg vereist dat zorgverleners hun communicatiestijl aanpassen aan de zorgvragers, omdat kwaliteitsvol sociaal contact essentieel is en een positieve invloed heeft op zowel de fysieke als mentale gezondheid en levenskwaliteit van zorgvragers.
Er is echter kritiek op de manier waarop communicatie vaak plaatsvindt tussen zorgverleners en zorgvragers. Deze communicatie wordt soms als ontoereikend en betuttelend beschouwd, waarbij zorgverleners specifiek voor dit onderzoek oudere patiënten, vaak een kinderlijke stijl hanteren, gekenmerkt door een overdreven toonhoogte, eenvoudige taal en het gebruik van verkleinwoorden en troetelnamen, een stijl die bekend staat als 'secondary babytalk'.

In 2023 vertoont de bevolkingspiramide van het Vlaamse Gewest het kenmerkende profiel van een vergrijzende bevolking, met een toenemend percentage van 65-plussers (21\%) ten opzichte van 2000 (17\%). Dit demografische verandering brengt nieuwe uitdagingen met zich mee in de zorgsector, met name wat betreft de communicatie tussen zorgverleners en zorgbehoevenden in deze leeftijdscategorie.

De vergrijzing van de bevolking leidt tot een toename in het aantal ouderen die zorg nodig hebben, vaak met specifieke communicatiebehoeften. Dit kan variëren van gehoorproblemen tot cognitieve uitdagingen. Hierin ligt het belang dit mijn onderzoek: Een applicatie ontwikkelen die kenmerken van ‘secondary babytalk’ kan identificeren aan de hand van bepaalde speech-to-text modellen die beschikbaar zijn op de markt. Deze modellen zullen met elkaar vergeleken worden aan de hand van specifieke criteria. Dit zal resulteren dat het meest accurate model verkozen kan worden
\autocite{Vlaanderen.be}.
\section{Artificiële Intelligentie en Automatische Spraakherkenning}

   - AI en ASR: Een uiteenzetting van de concepten, de verschillen, en hoe ze zich tot elkaar verhouden. 

- Elderspeak: Definitie, voorbeelden, en het belang van aangepaste communicatie met ouderen. 


\section{Communicatie in de Zorg}
   - Het belang van effectieve communicatie in de zorg, met een focus op ouderen. 

- De rol van AI en ASR in het verbeteren van communicatieprocessen
\section{Analyse van ASR-technologieën}
\subsection{Beschikbare Modellen en Selectiecriteria}
  - Een overzicht van de huidige ASR-modellen. 

- Criteria voor de selectie van bepaalde modellen voor gebruik in de zorgsector. 
\subsection{Evaluatiemethoden en metrics}
   - Inleiding tot Evaluatiemethoden: Het belang van evaluatie in ASR. 

- Word Error Rate (WER): Basisconcept en beperkingen. 

- Beyond WER: Geavanceerdere evaluatiemethoden voor een completere beoordeling. 

- Evaluatie van Eigennamen en Specifieke Datasets: Het belang van nauwkeurigheid bij eigennamen en specifieke terminologie. 

- Bias en Diversiteit in Datasets: Uitdagingen en oplossingen voor bias en het waarborgen van diversiteit.  
\subsection{Specifieke Uitdagingen in de Zorg}
   - Communicatie met personen met accenten, verschillende spreeksnelheden, en het gebruik van Algemeen Nederlands (AN) versus dialect. 

- De impact van deze variabelen op de effectiviteit van ASR-systemen. 



\section{Implicaties en Toepassingen}
 
\subsection{Toepassingen van AI en ASR in de Zorg}
- Praktische voorbeelden van hoe AI en ASR kunnen bijdragen aan betere zorg voor ouderen. 



\subsection{Persoonlijke Bijdrage en Toekomstvisie}

- Reflectie op hoe jouw onderzoek en bevindingen kunnen bijdragen aan de verbetering van de zorgsector en communicatie met ouderen. 

- Toekomstige ontwikkelingen en potentieel onderzoek. 


Dit hoofdstuk bevat je literatuurstudie. De inhoud gaat verder op de inleiding, maar zal het onderwerp van de bachelorproef *diepgaand* uitspitten. De bedoeling is dat de lezer na lezing van dit hoofdstuk helemaal op de hoogte is van de huidige stand van zaken (state-of-the-art) in het onderzoeksdomein. Iemand die niet vertrouwd is met het onderwerp, weet nu voldoende om de rest van het verhaal te kunnen volgen, zonder dat die er nog andere informatie moet over opzoeken \autocite{Pollefliet2011}.

Je verwijst bij elke bewering die je doet, vakterm die je introduceert, enz.\ naar je bronnen. In \LaTeX{} kan dat met het commando \texttt{$\backslash${textcite\{\}}} of \texttt{$\backslash${autocite\{\}}}. Als argument van het commando geef je de ``sleutel'' van een ``record'' in een bibliografische databank in het Bib\LaTeX{}-formaat (een tekstbestand). Als je expliciet naar de auteur verwijst in de zin (narratieve referentie), gebruik je \texttt{$\backslash${}textcite\{\}}. Soms is de auteursnaam niet expliciet een onderdeel van de zin, dan gebruik je \texttt{$\backslash${}autocite\{\}} (referentie tussen haakjes). Dit gebruik je bv.~bij een citaat, of om in het bijschrift van een overgenomen afbeelding, broncode, tabel, enz. te verwijzen naar de bron. In de volgende paragraaf een voorbeeld van elk.

\textcite{Knuth1998} schreef een van de standaardwerken over sorteer- en zoekalgoritmen. Experten zijn het erover eens dat cloud computing een interessante opportuniteit vormen, zowel voor gebruikers als voor dienstverleners op vlak van informatietechnologie~\autocite{Creeger2009}.

Let er ook op: het \texttt{cite}-commando voor de punt, dus binnen de zin. Je verwijst meteen naar een bron in de eerste zin die erop gebaseerd is, dus niet pas op het einde van een paragraaf.

\lipsum[7-20]
